\documentclass[journal=jctc,manuscript=article]{achemso}

%%%%%%%%%%%%%%%%%%%%%%%%%%%%%%%%%%%%%%%%%%%%%%%%%%%%%%%%%%%%%%%%%%%%%
%% Place any additional packages needed here.  Only include packages
%% which are essential, to avoid problems later.
%%%%%%%%%%%%%%%%%%%%%%%%%%%%%%%%%%%%%%%%%%%%%%%%%%%%%%%%%%%%%%%%%%%%%
\usepackage{chemformula} % Formula subscripts using \ch{}
\usepackage[T1]{fontenc} % Use modern font encodings

%%%%%%%%%%%%%%%%%%%%%%%%%%%%%%%%%%%%%%%%%%%%%%%%%%%%%%%%%%%%%%%%%%%%%
%% If issues arise when submitting your manuscript, you may want to
%% un-comment the next line.  This provides information on the
%% version of every file you have used.
%%%%%%%%%%%%%%%%%%%%%%%%%%%%%%%%%%%%%%%%%%%%%%%%%%%%%%%%%%%%%%%%%%%%%
%%\listfiles

%%%%%%%%%%%%%%%%%%%%%%%%%%%%%%%%%%%%%%%%%%%%%%%%%%%%%%%%%%%%%%%%%%%%%
%% Place any additional macros here.  Please use \newcommand* where
%% possible, and avoid layout-changing macros (which are not used
%% when typesetting).
%%%%%%%%%%%%%%%%%%%%%%%%%%%%%%%%%%%%%%%%%%%%%%%%%%%%%%%%%%%%%%%%%%%%%
% \newcommand*\mycommand[1]{\texttt{\emph{#1}}}

\usepackage{fullpage}
\usepackage{amsfonts}
\usepackage{graphicx}
\usepackage{float}
\usepackage{amsmath}
\usepackage{chemfig}
\usepackage{indentfirst}
\usepackage{longtable}
\usepackage{array}
\usepackage{cellspace}
\usepackage{palatino}
%\usepackage{breqn}
\usepackage{amssymb}
\usepackage{verbatim}
\usepackage[colorlinks=true,citecolor=blue,linkcolor=blue]{hyperref}
\usepackage{siunitx}
\usepackage{xr}
%\usepackage{bibentry}
\usepackage{verbatim}

%\DefineVerbatimEnvironment%
%	{verbatimprog}%
%	{Verbatim}%
%	{fontsize=\footnotesize}%

\newenvironment{myequation}{%
\addtocounter{equation}{-1}
\refstepcounter{defcounter}
\renewcommand\theequation{SI.\thedefcounter}
\begin{equation*}}
{\end{equation*}}

\renewcommand{\thefigure}{SI.\arabic{figure}}

\renewcommand{\thepage}{SI.\arabic{page}}

\renewcommand{\thesection}{SI.\Roman{section}}

\renewcommand{\thetable}{SI.\Roman{table}}

\makeatletter
\newcommand*{\addFileDependency}[1]{% argument=file name and extension
	\typeout{(#1)}
	\@addtofilelist{#1}
	\IfFileExists{#1}{}{\typeout{No file #1.}}
}
\makeatother

\newcommand*{\myexternaldocument}[1]{%
	\externaldocument{#1}%
	\addFileDependency{#1.tex}%
	\addFileDependency{#1.aux}%
}

\myexternaldocument{H:/Publications/Postdoc_3/VLE_PVT_UA_Mie_manuscript}

\SectionNumbersOn

% The figures are in a figures/ subdirectory.
\graphicspath{{figures/}}

%\bibliographystyle{apsrevlong}
%\bibliographystyle{apsrev}
\bibliographystyle{unsrt}

% italicized boldface for math (e.g. vectors)
\newcommand{\bfv}[1]{{\mbox{\boldmath{$#1$}}}}
% non-italicized boldface for math (e.g. matrices)
\newcommand{\bfm}[1]{{\bf #1}}          

%\newcommand{\bfm}[1]{{\mbox{\boldmath{$#1$}}}}
%\newcommand{\bfm}[1]{{\bf #1}}
\newcommand{\expect}[1]{\left \langle #1 \right \rangle} % <.> for denoting expectations over realizations of an experiment or thermal averages

\newcommand{\var}[1]{{\mathrm var}{(#1)}}
\newcommand{\x}{\bfv{x}}
\newcommand{\y}{\bfv{y}}
\newcommand{\f}{\bfv{f}}

\newcommand{\hatf}{\hat{f}}

\newcommand{\bTheta}{\bfm{\Theta}}
\newcommand{\btheta}{\bfm{\theta}}
\newcommand{\bhatf}{\bfm{\hat{f}}}
\newcommand{\Cov}[1] {\mathrm{cov}\left( #1 \right)}
\newcommand{\T}{\mathrm{T}}                                % T used in matrix transpose

\title{Supporting Information: Inaccurate extrapolation toward high pressures using united-atom, Mie $\lambda$-6 force fields parameterized with vapor-liquid equilibria properties.}

\author{Richard A. Messerly}
\email{richard.messerly@nist.gov}
\affiliation{Thermodynamics Research Center, National Institute of Standards and Technology, Boulder, Colorado, 80305}

\author{Michael R. Shirts}
\email{michael.shirts@colorado.edu}
\affiliation{Department of Chemical and Biological Engineering, University of Colorado, Boulder, Colorado, 80309}

\author{Andrei F. Kazakov}
\email{andrei.kazakov@nist.gov}
\affiliation{Thermodynamics Research Center, National Institute of Standards and Technology, Boulder, Colorado, 80305}

\keywords{Transferability, Molecular Dynamics, Molecular Simulation, Monte Carlo, Markov Chain, Bayesian Inference}%Use showkeys class option if keyword

\begin{document}

\section{Simulation Set-Up} \label{Simulation Set-Up}

This section is provided to improve the reproducibility of the results presented in this study.

\subsection{State Points} \label{State Points}

Tables \ref{tab:Ethane state points}, \ref{tab:C3H8 state points}, \ref{tab:C4H10 state points}, \ref{tab:C8H18 state points}, \ref{tab:IC4H10 state points}, \ref{tab:IC6H14 state points}, \ref{tab:IC8H18 state points}, and \ref{tab:NEOC5H12 state points} contain the state points that were simulated for ethane, propane, \textit{n}-butane, \textit{n}-octane, isobutane, isohexane, isooctane, and neopentane, respectively. The first 10 state points of each table correspond to five isochores while the last 9 points are for the supercritical isotherm. The number of state points, the specified reduced temperatures, and the spacing between neighboring densities were recommended by the developers of the ITIC approach (J. Richard Elliott and Seyed Mostafa Razavi). It has been demonstrated that these points are sufficient for accurate calculation of $\rho_{\rm l}^{\rm sat}$, $\rho_{\rm v}^{\rm sat}$, and $P_{\rm v}^{\rm sat}$ \cite{Mostafa_Diss}. Note that the temperatures $(T_{\rm sim})$, box lengths $(L_{\rm box})$, and number of molecules $(N_{\rm M})$ are the exact values used in simulation while the density $(\rho)$ is approximate (rounded) since it is calculated from $L_{\rm box}$, $N_{\rm M}$, and the molecular weight. 

\begin{table}[htb!]
	\caption{State points simulated for ethane.} \label{tab:Ethane state points}
	\begin{center}
		\begin{tabular}{|c|c|c|c|}
			\hline
			$T_{\rm sim}$ (K) & $L_{\rm box}$ (nm) & $N_{\rm M}$ (molecules) & $\rho \left(\frac{\rm kg}{\rm m^3}\right)$ \\ \hline
			137.0 & 3.21680 & 400 & 600.01 \\
			198.5 & 3.21680 & 400 & 600.01 \\ 
			174.0 & 3.29730 & 400 & 557.13 \\
			234.6 & 3.29730 & 400 & 557.13 \\
			207.0 & 3.38640 & 400 & 514.30 \\
			262.9 & 3.38640 & 400 & 514.30 \\
			236.0 & 3.48610 & 400 & 471.42 \\
			285.1 & 3.48610 & 400 & 471.42 \\
			260.0 & 3.59860 & 400 & 428.58 \\
			301.9 & 3.59860 & 400 & 428.58 \\
			360.0 & 6.15360 & 400 & 85.712 \\
			360.0 & 4.88410 & 400 & 171.43 \\
			360.0 & 4.26660 & 400 & 257.15 \\
			360.0 & 3.87650 & 400 & 342.85 \\
			360.0 & 3.59860 & 400 & 428.58 \\
			360.0 & 3.48610 & 400 & 471.42 \\
			360.0 & 3.38640 & 400 & 514.30 \\
			360.0 & 3.29730 & 400 & 557.13 \\
			360.0 & 3.21680 & 400 & 600.01 \\
			\hline
		\end{tabular}
	\end{center}
\end{table}

\begin{table}[htb!]
	\caption{State points simulated for propane.} \label{tab:C3H8 state points}
	\begin{center}
		\begin{tabular}{|c|c|c|c|}
			\hline
			$T_{\rm sim}$ (K) & $L_{\rm box}$ (nm) & $N_{\rm M}$ (molecules) & $\rho \left(\frac{\rm kg}{\rm m^3}\right)$ \\ \hline
			166 & 3.55643  & 400 & 651.13 \\
			242 & 3.55643  & 400 & 651.13 \\
			210 & 3.64538 & 400 & 604.62 \\
			285 & 3.64538 & 400 & 604.62 \\
			250 & 3.74395 & 400 & 558.11 \\
			320 & 3.74395 & 400 & 558.11 \\
			285 & 3.85413 & 400 & 511.60 \\
			347 & 3.85413 & 400 & 511.60 \\
			314 & 3.97854 & 400 & 465.09 \\
			368 & 3.97854 & 400 & 465.09 \\
			444 & 6.80321 & 400 & 93.019 \\
			444 & 5.39971 & 400 & 186.04 \\
			444 & 4.71708  & 400 & 279.06 \\
			444 & 4.28575 & 400 & 372.08 \\
			444 & 3.97854 & 400 & 465.09 \\
			444 & 3.85413 & 400 & 511.60 \\
			444 & 3.74395 & 400 & 558.11 \\
			444 & 3.64538 & 400 & 604.62 \\
			444 & 3.55643  & 400 & 651.13 \\
			\hline
		\end{tabular}
	\end{center}
\end{table}

\begin{table}[htb!]
	\caption{State points simulated for \textit{n}-butane.} \label{tab:C4H10 state points}
	\begin{center}
		\begin{tabular}{|c|c|c|c|}
			\hline
			$T_{\rm sim}$ (K) & $L_{\rm box}$ (nm) & $N_{\rm M}$ (molecules) & $\rho \left(\frac{\rm kg}{\rm m^3}\right)$ \\ \hline
			191 & 3.83864 & 400 & 682.53 \\
			278 & 3.83864 & 400 & 682.53 \\
			241 & 3.93465 & 400 & 633.78 \\
			327 & 3.93465 & 400 & 633.78 \\
			287 & 4.04104 & 400 & 585.03 \\
			367 & 4.04104 & 400 & 585.03 \\
			328 & 4.15997 & 400 & 536.28 \\
			399 & 4.15997 & 400 & 536.28 \\
			361 & 4.29425 & 400 & 487.52 \\
			423 & 4.29425 & 400 & 487.52 \\
			510 & 7.34306 & 400 & 97.50  \\
			510 & 5.82819 & 400 & 195.01 \\
			510 & 5.09140 & 400 & 292.51 \\
			510 & 4.62584 & 400 & 390.02 \\
			510 & 4.29425 & 400 & 487.52 \\
			510 & 4.15997 & 400 & 536.28 \\
			510 & 4.04104 & 400 & 585.03 \\
			510 & 3.93465 & 400 & 633.78 \\
			510 & 3.83864 & 400 & 682.53 \\
			\hline
		\end{tabular}
	\end{center}
\end{table}

\begin{table}[htb!]
	\caption{State points simulated for \textit{n}-octane.} \label{tab:C8H18 state points}
	\begin{center}
		\begin{tabular}{|c|c|c|c|}
			\hline
			$T_{\rm sim}$ (K) & $L_{\rm box}$ (nm) & $N_{\rm M}$ (molecules) & $\rho \left(\frac{\rm kg}{\rm m^3}\right)$ \\ \hline
			285.92 & 5.98449  & 800 & 708.01 \\
			387.29 & 5.98449  & 800 & 708.01 \\
			347.68 & 6.13416  & 800 & 657.44 \\
			440.25 & 6.13416  & 800 & 657.44 \\
			404.46 & 6.30003  & 800 & 606.87 \\
			483.20 & 6.30003  & 800 & 606.87 \\
			451.48 & 6.48542  & 800 & 556.30 \\
			515.25 & 6.48542  & 800 & 556.30 \\
			490.78 & 6.69481  & 800 & 505.72 \\
			539.92 & 6.69481  & 800 & 505.72 \\
			600.00 & 11.44803 & 800 & 101.14 \\
			600.00 & 9.08616  & 800 & 202.29 \\
			600.00 & 7.93753  & 800 & 303.43 \\
			600.00 & 7.21175  & 800 & 404.58 \\
			600.00 & 6.69481  & 800 & 505.72 \\
			600.00 & 6.48542  & 800 & 556.30 \\
			600.00 & 6.30003  & 800 & 606.87 \\
			600.00 & 6.13416  & 800 & 657.44 \\
			600.00 & 5.98449  & 800 & 708.01 \\
			\hline
		\end{tabular}
	\end{center}
\end{table}

\begin{table}[htb!]
	\caption{State points simulated for isobutane.} \label{tab:IC4H10 state points}
	\begin{center}
		\begin{tabular}{|c|c|c|c|}
			\hline
			$T_{\rm sim}$ (K) & $L_{\rm box}$ (nm) & $N_{\rm M}$ (molecules) & $\rho \left(\frac{\rm kg}{\rm m^3}\right)$ \\ \hline
			184 & 4.85814 & 800 & 673.40 \\
			267 & 4.85814 & 800 & 673.40 \\
			232 & 4.97964 & 800 & 625.30 \\
			315 & 4.97964 & 800 & 625.30 \\
			276 & 5.11429 & 800 & 577.20 \\
			353 & 5.11429 & 800 & 577.20 \\
			315 & 5.26480 & 800 & 529.10 \\
			383 & 5.26480 & 800 & 529.10 \\
			347 & 5.43475 & 800 & 481.00 \\
			406 & 5.43475 & 800 & 481.00 \\
			489 & 9.29328 & 800 & 96.20  \\
			489 & 7.37608 & 800 & 192.40 \\
			489 & 6.44360 & 800 & 288.60 \\
			489 & 5.85440 & 800 & 384.80 \\
			489 & 5.43475 & 800 & 481.00 \\
			489 & 5.26480 & 800 & 529.10 \\
			489 & 5.11429 & 800 & 577.20 \\
			489 & 4.97964 & 800 & 625.30 \\
			489 & 4.85814 & 800 & 673.40 \\
			\hline
		\end{tabular}
	\end{center}
\end{table}

\begin{table}[htb!]
	\caption{State points simulated for isohexane.} \label{tab:IC6H14 state points}
	\begin{center}
		\begin{tabular}{|c|c|c|c|}
			\hline
			$T_{\rm sim}$ (K) & $L_{\rm box}$ (nm) & $N_{\rm M}$ (molecules) & $\rho \left(\frac{\rm kg}{\rm m^3}\right)$ \\ \hline
			224 & 5.43297  & 800 & 713.86 \\
			326 & 5.43297  & 800 & 713.86 \\
			282 & 5.56885  & 800 & 662.87 \\
			383 & 5.56885  & 800 & 662.87 \\
			337 & 5.71943  & 800 & 611.88 \\
			431 & 5.71943  & 800 & 611.88 \\
			384 & 5.88774  & 800 & 560.89 \\
			467 & 5.88774  & 800 & 560.89 \\
			423 & 6.07780  & 800 & 509.90 \\
			495 & 6.07780  & 800 & 509.90 \\
			597 & 10.39289 & 800 & 101.98 \\
			597 & 8.24884  & 800 & 203.96 \\
			597 & 7.20603  & 800 & 305.94 \\
			597 & 6.54711  & 800 & 407.92 \\
			597 & 6.07780  & 800 & 509.90 \\
			597 & 5.88774  & 800 & 560.89 \\
			597 & 5.71943  & 800 & 611.88 \\
			597 & 5.56885  & 800 & 662.87 \\
			597 & 5.43297  & 800 & 713.86 \\
			\hline
		\end{tabular}
	\end{center}
\end{table}

\begin{table}[htb!]
	\caption{State points simulated for isooctane.} \label{tab:IC8H18 state points}
	\begin{center}
		\begin{tabular}{|c|c|c|c|}
			\hline
			$T_{\rm sim}$ (K) & $L_{\rm box}$ (nm) & $N_{\rm M}$ (molecules) & $\rho \left(\frac{\rm kg}{\rm m^3}\right)$ \\ \hline
			245 & 5.92132  & 800 & 730.91 \\
			356 & 5.92132  & 800 & 730.91 \\
			309 & 6.06941  & 800 & 678.71 \\
			419 & 6.06941  & 800 & 678.71 \\
			369 & 6.23353  & 800 & 626.50 \\
			472 & 6.23353  & 800 & 626.50 \\
			421 & 6.41697  & 800 & 574.29 \\
			512 & 6.41697  & 800 & 574.29 \\
			464 & 6.62411  & 800 & 522.08 \\
			543 & 6.62411  & 800 & 522.08 \\
			653 & 11.32707 & 800 & 104.42 \\
			653 & 8.99031  & 800 & 208.83 \\
			653 & 7.85376  & 800 & 313.25 \\
			653 & 7.13561  & 800 & 417.66 \\
			653 & 6.62411  & 800 & 522.08 \\
			653 & 6.41697  & 800 & 574.29 \\
			653 & 6.23353  & 800 & 626.50 \\
			653 & 6.06941  & 800 & 678.71 \\
			653 & 5.92132  & 800 & 730.91 \\
			\hline
		\end{tabular}
	\end{center}
\end{table}

\begin{table}[htb!]
	\caption{State points simulated for neopentane.} \label{tab:NEOC5H12 state points}
	\begin{center}
		\begin{tabular}{|c|c|c|c|}
			\hline
			$T_{\rm sim}$ (K) & $L_{\rm box}$ (nm) & $N_{\rm M}$ (molecules) & $\rho \left(\frac{\rm kg}{\rm m^3}\right)$ \\ \hline
			257 & 5.34568  & 800 & 627.43 \\
			344 & 5.34568  & 800 & 627.43 \\
			300 & 5.47938  & 800 & 582.61 \\
			380 & 5.47938  & 800 & 582.61 \\
			337 & 5.62754  & 800 & 537.79 \\
			409 & 5.62754  & 800 & 537.79 \\
			368 & 5.79315  & 800 & 492.98 \\
			431 & 5.79315  & 800 & 492.98 \\
			393 & 5.98015  & 800 & 448.16 \\
			448 & 5.98015  & 800 & 448.16 \\
			520 & 10.22592 & 800 & 89.63  \\
			520 & 8.11632  & 800 & 179.26 \\
			520 & 7.09026  & 800 & 268.90 \\
			520 & 6.44193  & 800 & 358.53 \\
			520 & 5.98015  & 800 & 448.16 \\
			520 & 5.79315  & 800 & 492.98 \\
			520 & 5.62754  & 800 & 537.79 \\
			520 & 5.47938  & 800 & 582.61 \\
			520 & 5.34568  & 800 & 627.43 \\
			\hline
		\end{tabular}
	\end{center}
\end{table}

\subsection{GROMACS Input Files} \label{GROMACS Input Files}

We have provided example input files for simulating isooctane at 653.0 K with the TraPPE-UA force field in GROMACS (see attached .gro, .top, and .mdp files).

\bibliography{postdoc_references}

%\nobibliography{H:/Publications/Postdoc_3/postdoc_references.bib}{}

\end{document}