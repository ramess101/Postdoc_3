\documentclass[preprint,letterpaper,floatfix,citeautoscript,aip,jcp]{revtex4-1}

\usepackage{fullpage}
\usepackage{amsfonts}
\usepackage{graphicx}
\usepackage{amsmath}
\usepackage{chemfig}
\usepackage{indentfirst}

\begin{document}

\title{Bayesian inference analysis of transferable, united-atom, Mie $\lambda$-6 force fields for normal and branched alkanes. To be submitted to the Journal of Physical Chemistry, B.}
%Alternative titles: Impact of repulsive exponent on high pressure systems
%MRS1: suggestion: make the title the conclusion?  Such as ``Baysian inference shows that Mie-6 united atom force fields cannot predict vapor-liquid equilibria for alkanes?''.     
\author{Richard A. Messerly}
\email{richard.messerly@nist.gov}
\affiliation{Thermodynamics Research Center, National Institute of Standards and Technology, Boulder, Colorado, 80305}

\keywords{Transferability, Force Field, Molecular Dynamics, Molecular Simulation}%Use showkeys class option if keyword

\begin{abstract}

\end{abstract}

\maketitle

\section*{Purpose}

The aim of this study is to demonstrate, using Bayesian inference, that a UA Mie force field cannot adequately predict VLE and compressed liquid pressures for normal and branched alkanes. For adequate prediction of VLE and compressed liquid pressures, we recommend using AUA models or perhaps Exp-6 or extended Lennard-Jones, i.e. 12-10-8-6. %We then use simple Bayes factors (if not RJMC) to determine the optimal value of lambda for predicting compressed liquid pressures. %%%% RAM: Andrei and I think that we don't really need to find the best lambda value since most people would not be willing to sacrifice Pvsat just to match high pressures. Also Andrei wants to minimize the amount of effort/time required to get this paper ready.

\section*{Outline}

%%% These were some of my thoughts before meeting with Andrei. Skip this section.

%\begin{enumerate}
%	\item Introduction
%	\item Bayesian Theory
%	\begin{enumerate}
%		\item Posterior
%		\item MCMC
%		\item RJMC
%	\end{enumerate}
%    \item MBAR-ITIC
%    \item Simulation details
%    \item Force field
%	\item Lambda (RJMC)
%	\item Transferability of CH2 sites
%	\item Transferability of CH3 sites using RJMC
%	\item Combining rules (RJMC)
%	\item Viscosity PoU
%	\item Higher pressure PoU
%\end{enumerate}
%
%Are we justified in picking just a single value for lambda? Or should we have a range of lambda values?
%Are we justified in fitting CH3 and CH2 sites simultaneously?
%Is one combining rule favored over another?
%
%I think for this paper we could focus on just a couple of these points. RJMC needs to be working for some of these. Lambda I could probably do without RJMC. Same with CH2. But others it would be necessary. 
%I basically already have the results for lambda (just ethane) and for the CH2 sites.
%
%Oh, for CH2 I could use RJMC to see if we should have a single CH2 site or if we need to have a CH2 for each. Basically I would have the same posterior (combined for all three) and I would have the model choose between the same and different. Right?
%
%So if I develop surrogate models for logp, I could perform this analysis very quickly. Then if I develop them for rhol, Psat I can change the likelihood function. For now though I am just going to use logp.
%
%Several decisions are made somewhat arbitrarily when developing force fields...
%
%I can perform a simple Bayesian inference analysis and show how the Mie potential cannot accurately predict VLE and high pressure PVT
%I could just perform this analysis for the 16-6 potential, since I already have that all done for n-alkanes
%I can show how TraPPE and Potoff perform to show the opposite trends of the 12-6 and 16-6
%Then show how the uncertainty in epsilon and sigma cannot reconcile this for the 16-6
%Then show how even a 15-6 or 14-6 potential cannot match all three properties
%
%The goal of this study is to determine if there exists a set of eps, sig, lam that accurately predicts VLE and supercritical PVT
%Specifically, we are testing whether or not the UA approach can adequately predict high pressure densities
%
%We can show how TraPPE-UA2 and/or TAMie does a better job at high pressures
%I could also simulate Exp-6

\begin{enumerate}
	\item Introduction
	\item Methods I
	\begin{enumerate}
		\item Simulation details
		\item Force field
	\end{enumerate}
	\item Case Study for alkanes
	\item Methods II
	\begin{enumerate}
		\item Bayesian Analysis
		\item Surrogate Model
		\item Propagation of Uncertainty
		%%%% RAM: Do we want to consider performing a Pareto front analysis to show that no set of eps, sig, lam can match all three? %%%%
	\end{enumerate}
	\item Results
	\begin{enumerate}
		\item VLE and Compressed
		\begin{enumerate}
			\item Parameter uncertainties
			\item Propagation of uncertainties
		\end{enumerate}
	    %\item Optimal $\lambda$ for high pressures %%% RAM: This distracts from the main purpose, that the Mie potential cannot match all three properties
	\end{enumerate}
	\item Future Work
	\item Conclusions	
\end{enumerate}

%	\item Higher pressure PoU
%	\item Different values of lambda cannot reconcile (already done for ethane, probably need to do for other alkanes)
%	\item AUA, AUA Mie, Exp-6

\section*{Detailed Outline}

\section{Introduction}

\begin{enumerate}
	\item Developing reliable fundamental equations of state (REFPROP) is an arduous task that relies on having high accuracy data over a wide range of state points
	\item Reliable predictions for high pressure systems are important for many industrial applications
	\item Recently, molecular simulation results have supplemented experimental data when developing fundamental equations of state
	\item In addition, the 10th Industrial Fluid Properties Simulation Challenge is to predict the viscosities at high pressures of a highly branched alkane. Reliable PVT predictions are essential for this challenge.
	\item The UA Mie potential has received significant attention for its ability to predict VLE without requiring an all-atom representation
	\item However, the impact that modifying the repulsive exponent has on the higher pressure states has not been tested
	\item The purpose of this study is to perform a rigorous Bayesian analysis as to the adequacy of a UA Mie potential for predicting both VLE and compressed liquid/supercritical pressures
\end{enumerate}


\section{Methods I}

\subsection{Simulation Details}

\begin{enumerate}
	\item The compounds simulated in this study are ethane, propane, \textit{n}-butane, \textit{n}-octane, isobutane, isopropane, isohexane, isooctane, and neopentane
	\item These compounds were selected as a sample set for 2,2,4-trimethylhexane, the compound studied in the 10th Industrial Fluid Properties Simulation Challenge
	\item We perform NVT simulations in GROMACS along a supercritical isotherm and five isochores that correspond to saturated liquid densities
	\item We use ITIC to convert the Udep and Z values obtained from the NVT simulations to rholsat and Pvsat
	\item The specific state points are provided in supporting information
	\item The GROMACS settings are provided in supporting information
\end{enumerate}

\subsection{Force field}

\begin{enumerate}
	\item For each compound studied, we use a united-atom representation as defined by the TraPPE (and Mie) force fields. For normal and branched alkanes this consists of CH3, CH2, CH, and C sites.
	\item We use fixed bond-lengths, harmonic angles, cosine series for torsions, and exclude 1-4 nonbonded interactions 
	\item Bonded parameters are provided in supporting information
	\item Nonbonded interactions for intermolecular and intramolecular sites separated by more than 3 bonds are represented using a Mie potential, which can be viewed as a generalized Lennard-Jones
	\item (Maybe include Exp-6 and extended Lennard-Jones, i.e. 12-10-8-6)
\end{enumerate}

\section{Case Study for alkanes}

\begin{enumerate}
	\item Several force fields in the literature have been optimized to agree with VLE properties (TraPPE, Potoff, TraPPE-2, TAMie, Errington, AUA4)
	\item TraPPE, Potoff, and AUA4 have parameters for each compound studied, while TraPPE-2 only has parameters for ethane, TAMie has parameters for all except isooctane and neopentane (containing a C group), and ErrExp-6 only has parameters for the \textit{n}-alkanes
	\item Figure: Z vs 1000/T for ethane, propane, n-butane, and n-octane
	\item Figure: Z vs 1000/T for isobutane, isopropane, isohexane, isooctane and neopentane 
	\item PVT (Z) trends are inaccurate for both TraPPE and Potoff at high pressures, i.e. non-VLE conditions. Specifically, the TraPPE 12-6 under predicts while 16-6 over predicts
	\item Although these results might suggest that a 14-6 potential would work best, recall that the TraPPE 12-6 does not accurately predict Pvsat. 
	\item Since TraPPE and Potoff use slightly different objective functions we want to perform an equivalent analysis for different values of lambda
	\item Hypothesis that we want to test is that there does not exist a set of epsilon, sigma, and lambda that provides reasonable VLE, supercritical fluid, and compressed liquid (for ethane I already know this is not feasible. For n-alkanes I know that the 16-6 cannot accomplish this, but I am not sure about 14-6 or 15-6.)
	%%% RAM: Do we want to include results for the AUA models in the case study, future work, or supporting information?
	\item AUA LJ 12-6 is much more accurate for ethane
	\item AUA Mie 14-6 potential is not much better than UA Mie 16-6 for n-alkanes
	\item AUA Exp-6 force field is not much better than UA Mie 16-6 for n-alkanes
	\item No results for AUA4
\end{enumerate}

\section{Methods II}

\begin{enumerate}
	\item We use Bayesian inference with MCMC to quantify the uncertainty in the force field parameters
	%%%% RAM: Do we even need to perform an MCMC analysis? It is kind of overkill. We could just plot the credible regions by scanning the parameter space.
	\item We perform direct molecular simulation for several reference force field parameter sets
	\item We use MBAR-ITIC to predict rholsat and Pvsat with non-simulated force field parameters
	\item We use MBAR to propagate the parameter uncertainties from VLE to Z of compressed liquids/supercritical
	\item We perform this analysis for CH3 and CH2 sequentially and independently. In other words, the Markov Chain only samples a 2-dimensional space. % RAM: Andrei and I feel like doing this for CH and C is more work than necessary
\end{enumerate}

\subsection{Bayesian Analysis}

\begin{enumerate}
	\item By quantifying the uncertainty in epsilon and sigma for a given lambda, we can determine if the Mie potential is adequate for reproducing VLE and compressed liquid/supercritical pressures
	\item Posterior includes saturated liquid density and vapor pressure
	%%% RAM: Or should we compare several different posteriors? I.e. include rholsat and Pvsat, or rholsat and highP, or all three?
	\item Markov Chain Monte Carlo is used to sample % RAM: Again, do we need to do MCMC or can we simplify things by just performing a 2-D scan
	\item Details for MCMC are provided in supporting information (i.e. number of steps for burn-in and production, frequency that step sizes are updated, resulting acceptance percentages, etc.)
	\item The parameter uncertainty is propagated when predicting high pressures
	% RAM:  No longer finding the optimal lambda for high pressures \item To determine the optimal value of lambda for compressed liquid pressures, we redefine the posterior using only saturated liquid density and compressed liquid pressures  
\end{enumerate}

\subsection{Surrogate Model}

\begin{enumerate}
	\item A Bayesian analysis is computationally too expensive if direct molecular simulations are performed for every MCMC step
	\item As demonstrated in a previous publication, MBAR can reweight the configurations that are sampled from different force fields without direct simulation
	\item In other words, a set of reference force fields are simulated for each molecule and MBAR is used instead of direct simulation for each MCMC step
	\item As demonstrated in a previous publication, MBAR is used to predict Udep and Z while ITIC is used to convert Udep and Z to rholsat and Pvsat
	\item ITIC state points are fit to rectilinear and Antoine equation to interpolate rholsat and Pvsat, this allows for comparison with experimental data at hundreds of temperatures
	\item The likelihood includes the experimental uncertainties but, more importantly, the numerical uncertainties. In other words, the numerical uncertainties account for the uncertainties that arise from the simulations themselves, the MBAR reweighting, the ITIC algorithm, and fitting to rectilinear and Antoine.
	\item To leave no room for doubt in our conclusions, we use very conservative (and empirical) estimates of numerical uncertainty for rholsat and Pvsat (see Supporting Information)
\end{enumerate}

\subsection{Propagation of Uncertainty}

\begin{enumerate}
	\item From the MCMC parameter sets, we randomly sample a subset of 100-1000 parameter sets
	\item We use MBAR to predict Z for each of those parameter sets
	\item We plot these results as a histogram to determine the 95\% credible interval for Z at each state point
\end{enumerate}

\section{Results}

\subsection{VLE and Compressed}

\subsubsection{Parameter uncertainties}

\begin{enumerate}
	\item Figure: The uncertainty regions for CH3, CH2, CH, and C. I can include 14-6, 15-6, and 16-6. Perhaps I will only do this rigorous analysis for CH3 or for CH3 and CH2. Probably not for all. I could include the results from the alternative posterior (excluding Pvsat and including high pressures) but then it might be out of place in this section.
	\item Bayes factors demonstrate that, for VLE, a 15-6 or 16-6 potential are favored significantly more than a 14-6 (could include 17-6 or 18-6 as well)
	\item CH2 credible regions overlap considerably between propane, n-butane, and n-octane
	\item By comparing the Bayes factor of a transferable CH2 site and three independent CH2 sites we observe that the CH2 sites are indistinguishable
	\item Statement about CH credible regions for isobutane, isopentane, and isohexane
	\item Statement about C credible region for isooctane and neopentane
\end{enumerate}

\subsubsection{Propagation of uncertainties}

\begin{enumerate}
	\item Figure: Uncertainties in rholsat and Pvsat for n-alkanes. Include 14-6, 15-6, 16-6.
	\item Figure: Uncertainties in rholsat and Pvsat for branched alkanes. Include only 16-6.
	\item Clearly the uncertainties are fairly conservative due to the relatively large numerical uncertainties we assigned in the posterior
	\item Figure: Z vs 1000/T for n-alkanes where the error bars represent the Bayesian uncertainties from VLE. Include 14-6, 15-6, and 16-6 results.
	\item The 16-6 potential is not able to predict both VLE and compressed liquid/supercritical pressures
	\item VLE is much worse for 14-6, about the same for 15-6
	\item Condensed liquid pressures are slightly better for 14-6 and 15-6 but still over predict
	\item Figure: Z vs 1000/T for branched alkanes. Results are only included for the 16-6 potential.
	\item Same results as for normal alkanes.
\end{enumerate}

%%% RAM: We feel like this discussion could just be a paragraph saying that if Pvsat is not important you could use a 14-6 potential and match just liquid phase pressures
%\subsection{Optimal $\lambda$ for high pressures}

\begin{enumerate}
	\item We modify the posterior by excluding the Pvsat data and including the REFPROP correlations at high pressures
	\item Figure: I can either include the parameter uncertainties here or back in the Parameter Uncertainties section. I could even move this to supporting information
	\item Bayes ratios show the evidence for different values of $\lambda$
	\item We recommend that lower values of $\lambda$ be favored
\end{enumerate}

\section{Future Work}

\begin{enumerate}
	\item As observed in the case study, the AUA approach typically provides more reliable estimates at high pressures.
	\item At higher pressures you need the hydrogens, the higher the shift in the bond-length the better.
	\item For example, notice that the AUA LJ (TraPPE-2) model is better than the AUA Mie (TAMie) and AUA Exp-6 (ErrExp-6), despite having fewer fitting parameters. This is because they use a much larger bond displacement.
	%%% RAM: I do not have results for AUA4, do I really need all of these examples from the literature?	
	\item An alternative method to AUA is to use an extended Lennard-Jones potential, 12-10-8-6, that has the flexibility of a Mie potential but without the steep barrier
\end{enumerate}
 
\section{Conclusions}

\begin{enumerate}
	\item Although the UA Mie potential provides great improvement over the UA LJ 12-6 potential for VLE, it drastically over predicts pressures for supercritical fluids and compressed liquids (actually for supercritical fluids it over predicts at high densities and under predicts at low densities. Not sure I want to open up that can of worms.)
	\item By performing a rigorous statistical analysis, we verify that no set of $\epsilon$ and $\sigma$ can adequately predict VLE and high pressures (I need a better way to refer to this) for a 16-6, 15-6, or 14-6 potential.
\end{enumerate}

%%% This was taken from previous drafts of the first manuscript. These paragraphs will probably not go in this manuscript.
%%%% The next few paragraphs probably belong in Part II where we focus on parameterization. I think it suffices in this paper to just say that we need surrogate models for computational reasons.

%%% Some of this discussion actually belongs in the publication that I do with Elliott and Potoff most likely

%The increase in the number of model parameters causes the parameterization to be more difficult, especially when direct molecular simulations are required. For example, the Mie $\lambda$-6 parameters reported by Potoff are obtained by scanning the parameter space using predefined grid spacing. Although this scanning approach is useful for verifying that a global minimum is found, it scales as $O(n_g^{n_p})$ where $n_g$ is the number of grid points per $n_p$ and $n_p$ is the number of parameters. With $n_g \approx 30$ performing molecular simulations at each grid point becomes computationally intractable for $n_p > 3$. This is also problematic for performing a Pareto front \cite{Pareto_Deriv,Pareto_LJPQ,Pareto_ST} or feasible region \cite{Mess4} analysis that typically require a very refined grid of the parameter space. Furthermore, Bayesian methods that use Markov Chain Monte Carlo (MCMC) to sample from the parameter space become extremely expensive in higher dimensions when direct simulations are required at each step \cite{Bay_UQ,Bay_Deriv,Bay_MD}. 
%%% This may not be true. Higher dimensional optimizations actually are less likely to get trapped, apparently.
%A common problem for any high dimensional parameter space is that gradient based optimizations can get trapped in local minima while so-called global optimizations may require inordinate number of ``function evaluations'' (i.e. molecular simulations). Increasing the number of parameters can also lead to a high degree of parameter correlation. In addition, over-parameterization can result in non-physical optimal parameters which will likely extrapolate poorly. For these reasons, it is common to make model simplifications by reducing the number of parameters in a judicious manner. There are four primary ways to accomplish this: 1) optimizing the intramolecular contribution independent of the intermolecular potential 2) constraining parameters in the non-bonded potential 3) employing combining rules and 4) transferring parameters for similar site types. Unfortunately, each of these model simplifications can lead to model deficiencies.
%
%Typically, intramolecular potentials are obtained by regressing model parameters to match quantum mechanical calculations of different configurations. Subsequently, the intermolecular (non-bonded) potentials are often fit to reproduce experimental data, such as saturated liquid density, vapor pressure, and heat of vaporization. It is commonly assumed that the uncertainty propagated from the intramolecular potential to the vapor-liquid equilibria properties $(\rho_{\rm l}^{\rm sat}, \rho_{\rm v}^{\rm sat}, P_{\rm v}^{\rm sat})$ is negligible relative to the uncertainty caused by the non-bonded potential \cite{Intra_Potoff,Mess4}. Therefore, recent studies that have reported high accuracy force fields have focused primarily on the non-bonded potential (with the main exception being the focus given to anisotropic-united-atoms for terminal sites). For this reason, the present study focuses on parameterizing the non-bonded potential. 
%
%Although the generalized $\lambda_{\rm rep}$--$\lambda_{disp}$ Mie potential (where $\lambda_{\rm rep}$ is the repulsive exponent and $\lambda_{disp}$ is the dispersive exponent) can use any floating point value for $\lambda_{\rm rep}$ and $\lambda_{disp}$, the common practice is to set $\lambda_{disp}=6$. The dispersive tail having an $r^{-6}$ dependence is well founded and should thus lead to improved extrapolation \cite{Mie}. In addition, it is common to only consider integer values of $\lambda_{\rm rep}$ (and sometimes only even integers). This has a nice computational advantage since it is much less expensive to compute a number raised to an integer power than a floating point power \cite{Mie}. Furthermore, this simplifies the optimization to a set of two-dimensional parameter spaces (in $\epsilon$ and $\sigma$) rather than a single three-dimensional parameter space (in $\epsilon, \sigma$, and $\lambda_{\rm rep}$). Unfortunately, this assumption reduces the model flexibility which may lead to inadequate representation of the target variables \cite{Avendano2013}. For example, Papaioannou et al. demonstrated that in many cases the optimal repulsive exponent $(\lambda_{\rm rep})$ is not an integer value \cite{Papaioannou2016}.
%
%Another way to reduce the number of model parameters is by implementing Lorentz-Berthelot (or some other form of) combining rules for cross-interactions. Cross-interaction parameters are the non-bonded parameters for two different site types. Combining rules reduce the number of fitting parameters from being $O(n_i^2)$ to $O(n_i)$, where $n_i$ is the number of site types. In addition, combining rules are intended to ensure that cross-interaction parameters are physically reasonable. That being said, the Lorentz-Berthelot combining rules have been called into question, especially for mixtures \cite{Delhommelle2001}. For this reason, many other \textit{ad hoc} combining rules have been proposed \cite{TraPPEUA2}.
%
%Finally, transferability is an essential assumption in molecular simulation. Transferability assumes that the non-bonded interactions are the same when two chemical moieties are in a similar environment, e.g. the CH$_2$ groups in \textit{n}-butane are the same as the CH$_2$ groups in \textit{n}-pentane. The assumption of transferability has been fundamental to force field development as it allows for a systematic sequential parameterization of functional groups. With this assumption a new chemical moiety is included in the force field by assuming all previous parameters are constant. Therefore, parameterizing the n$^{th}$ site type has the same cost as the first. Unfortunately, validation of transferability is an essential but difficult (and typically omitted) step. 
%
%The primary reason for this is that if two sites are found to not be transferable it may require significant reparameterization of previously optimized site types. For example, the improved TraPPE-UA2 CH$_3$ LJ parameters will likely necessitate reparameterization of other site types from the TraPPE-UA force field that were optimized using the previous TraPPE-UA CH$_3$ LJ parameters. This is an arduous and time-consuming process.
%
%Although the aforementioned simplifications can dramatically reduce the number of fitting parameters, it is not clear \textit{a priori} if these assumptions lead to model inadequacies. In fact, it is almost certain that they do. Ideally, it would be possible to optimize a force field to a large number of data and compounds simultaneously (rather than sequentially) and use rigorous statistical methods to select the optimal non-bonded potentials, validate combining rules, and determine when two sites are in fact transferable. This would require advanced high dimensional optimization routines such as genetic algorithms, leapfrog \cite{RHINEHART2012}, or Bayesian optimizers (see Ucyigitler et al. \cite{SPEADMD}). Unfortunately, these methods are not feasible when molecular simulations are performed at each step of the optimization algorithm. By contrast, Papaioannou et al. and Elliott et al. demonstrated how large amounts of compounds and data can be optimized simultaneously when using less expensive approaches, namely, the SAFT-$\gamma$ equation-of-state and SPEADMD, respectively \cite{Papaioannou2016,SPEADMD}. These methods allowed the authors to determine when additional parameters were needed to distinguish between site types and when two site types were considered indistinguishable.

%% SPEADMD is probably best classified as a configuration sampling based surrogate model

\end{document}
